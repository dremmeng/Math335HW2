% LaTeX Article Template - customizing page format
%
% LaTeX document uses 10-point fonts by default.  To use
% 11-point or 12-point fonts, use \documentclass[11pt]{article}
% or \documentclass[12pt]{article}.
\documentclass{article}

% Set left margin - The default is 1 inch, so the following 
% command sets a 1.25-inch left margin.
\setlength{\oddsidemargin}{0.25in}

% Set width of the text - What is left will be the right margin.
% In this case, right margin is 8.5in - 1.25in - 6in = 1.25in.
\setlength{\textwidth}{6in}

% Set top margin - The default is 1 inch, so the following 
% command sets a 0.75-inch top margin.
\setlength{\topmargin}{-0.25in}

% Set height of the text - What is left will be the bottom margin.
% In this case, bottom margin is 11in - 0.75in - 9.5in = 0.75in
\setlength{\textheight}{8in}
\usepackage{fancyhdr}
\usepackage{float}
\usepackage{mathtools}
\usepackage{amsmath}
\usepackage{amssymb}
\setlength{\parskip}{5pt} 
\pagestyle{fancyplain}
% Set the beginning of a LaTeX document
\begin{document}

\lhead{Drew Remmenga MATH 335}
\rhead{HW \#2}
%\lhead{Independent Study}
%\rhead{R Lab}

\begin{enumerate}

\item 

\begin{equation}
MSE[\hat{\theta}] = V[\hat{\theta}] + Bias[\hat{\theta}]^{2} \\
\end{equation}
\begin{equation}
Var[X] = E[X^{2}] - E[X]^{2} \\
\end{equation}
\begin{enumerate}
\item 
\begin{equation*}
\begin{split}
\hat{\theta} &= \frac{Y_{1} + 2 Y_{2}}{3} \\
\hat{\theta} &= \frac{Y_{1}}{3} + \frac{2 Y_{2}}{3} \\
E[\hat{\theta}] &= E[\frac{Y_{1}}{3} + \frac{2 Y_{2}}{3}] \\
E[\hat{\theta}] &=\frac{\theta}{3} + \frac{2\theta}{3}] \\
E[\hat{\theta}] &= \theta \\
Bias[\hat{\theta}] &= 0 \\
Var[\hat{\theta}] &= Var[\frac{Y_{1}}{3} + \frac{2 Y_{2}}{3}] \\
Var[\hat{\theta}] &= \frac{1}{9} Var[Y_{1} + 2 Y_{2}] \\
Var[\hat{\theta}] &= \frac{1}{9} (\theta^{2}+ 4 \theta^{2}] \\
Var[\hat{\theta}] &= \frac{5\theta^{2}}{9}
\end{split}
\end{equation*}
By (1):
\begin{equation*}
\begin{split}
MSE[\hat{\theta}] &= \frac{5\theta^{2}}{9} + 0^{2} \\
MSE[\hat{\theta}] &= \frac{5\theta^{2}}{9}
\end{split}
\end{equation*}
\item
\begin{equation*}
\begin{split}
\hat{\theta} &= \bar{Y} \\
E[\hat{\theta}] &= \theta \\
Bias[\hat{\theta}] &= 0 \\
Var[\hat{\theta}] &= \frac{\theta^{2}}{3}
\end{split}
\end{equation*}
By (1):
\begin{equation*}
\begin{split}
MSE[\hat{\theta}] &= \frac{\theta^{2}}{3} + 0^{2} \\
MSE[\hat{\theta}] &= \frac{\theta^{2}}{3}
\end{split}
\end{equation*}
\item
\begin{equation*}
\begin{split}
\hat{\theta} &= Y_{1}^{2}\\
E[\hat{\theta}] &= E[Y_{1}^{2}]\\
\end{split}
\end{equation*}
By (2):
\begin{equation*}
\begin{split}
E[\hat{\theta}] &= E[Y_{1}]^{2}+Var[Y_{1}]\\
E[\hat{\theta}] &=  \theta^{2}+\theta^{2} \\
E[\hat{\theta}] &=  2\theta^{2}\\
\end{split}
\end{equation*}
Then for the Variance:
\begin{equation*}
\begin{split}
Var[\hat{\theta}] &= (Y_{1}^{2} - \theta)^{2}P(Y=Y_{1}^{2}) \\
Var[\hat{\theta}] &= (Y_{1}^{2} - \theta)^{2}\frac{e^\frac{-Y_{1}^{2}}{\theta}}{\theta}
\end{split}
\end{equation*}
\end{enumerate}
\item
\begin{enumerate}
\item
$
\hat{\lambda} = \frac{\bar{X}}{8} \\
$
It is unbiased since we should expect this to be poisson with the same mean every hour of every work day. 
\item
\begin{equation*}
\begin{split}
E[C] &= E[50X + 2X^{2}] \\
E[C] &= 50E[X] + 2E[X^{2}] \\
\end{split}
\end{equation*}
By (2):
\begin{equation*}
\begin{split}
E[C] &= 50*8*\lambda + 2(V[X]+E[X]^{2}) \\
E[C] &= 400*\lambda + 2(V[X]+E[X]^{2}) \\
E[C] &= 400\lambda + 2(8*\lambda + (8*\lambda)^{2})\\
E[C] &= 416\lambda+128\lambda^{2}
\end{split}
\end{equation*}
\end{enumerate}
\item
\begin{enumerate}
\item
\begin{equation*}
\begin{split}
E[\bar{Y}] = a-\frac{1}{2} \\
Bias[\bar{Y}] = -\frac{1}{2}
\end{split}
\end{equation*}
\item
\begin{equation*}
\begin{split}
\hat{a} = \bar{Y} + \frac{1}{2}
\end{split}
\end{equation*}
\item
\begin{equation*}
\begin{split}
Var[\bar{Y}] &= \frac{1}{12} \\
MSE[\bar{Y}] &= \frac{1}{12} + (-\frac{1}{2})^{2} \\
MSE[\bar{Y}] &= \frac{1}{12} + \frac{1}{4} \\
MSE[\bar{Y}] &= \frac{4}{12} \\
MSE[\bar{Y}] &= \frac{1}{3}
\end{split}
\end{equation*}
\end{enumerate}
\item
\begin{equation*}
\begin{split}
Bias[\hat{p_{1}}] &= 0 \\
Var[\hat{p_{1}}] &= \frac{p(1-p)}{n} \\
MSE[\hat{p_{1}}] &= \frac{p(1-p)}{n} \\
\end{split}
\end{equation*}
\begin{equation*}
\begin{split}
Bias[\hat{p_{2}}] &= \frac{np+1}{n+2} - p \\
&= \frac{np+1}{n+2} - \frac{np + 2p}{n+2} \\
&= \frac{1-2p}{n+2} \\
Var[\hat{p_{2}} & = \frac{np(1-p)}{(n+2)^{2}}+\frac{1}{(n+2)^{2}} \\
MSE[\hat{p_{2}} &= \frac{np(1-p)+1}{(n+2)^{2}}+(\frac{1-2p}{n+2})^{2} \\
\end{split}
\end{equation*}
\end{enumerate}
Simulation
\begin{enumerate}
\item
.000292\\
.2547849
\item
1.5 and .255\\
Simulated vales are very close\\
1.500292 and .25478486\\
\item
.05287404 \\
.06467135


\item
1 and .05 \\
1.05287404 and .0618756832
\end{enumerate}


\end{document}
